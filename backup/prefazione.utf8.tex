%
% prefazione.tex
% 
\Preface
% la macro \Preface inserisce i seguenti comandi, commentati a seguire come '%p%'
% usare ques in modo da comporre correttamente la numerazione di pagina araba.
% se non si vuole scrivere la prefazione, allora inserire i comandi
% \chapter{preambolo} \prefacePageNumbering
%   oppure
% \chapter*{un preambolo, (pagine in romano)}
% \addcontentsline{toc}{chapter}{Il primo capitolo, non numerato} 
% \prefacePageNumbering 
%
%p%\chapter*{Prefazione} % se language=en, \chapter*{Preface}!
%p% nota: NON cambiare le righe seguenti --------------
%p% aggiunge 'Prefazione' all'indice, ma non numera 'prefazione'
%p% riallinea la numerazione di pagina, in funzione della presenza o meno 
%p% di pagine bianche iniziali (copertina fornita a parte dall'editore)
%p%\addcontentsline{toc}{chapter}{Prefazione}
%p%\prefacePageNumbering
% ----------------------------------------------------
%
% iniziare a scrivere qui sotto
Qui scrivere la prefazione. \\
\begin{flushright}
{\emph{L'autore.}}
\end{flushright}



% ---------------------------------------------------------------------
\chapter*{Sommario} \addcontentsline{toc}{chapter}{Sommario}
%% \newpage \vspace*{1cm} \section*{\Huge Piano dell'opera} \vspace*{1cm}
Qui scrivere il sommario, se piace, altrimenti cancellare, o commentare questa e le righe precedenti (fino alla riga di commento con tanti caratteri '-').\\
